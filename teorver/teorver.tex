%%% page margins
\documentclass[a4paper,12pt]{article}
\usepackage[margin=2.5cm]{geometry} % fix page margins

% strengthen samepage because penalties are unreliable
\newenvironment{onsamepage} {\begin{minipage}{\textwidth}} {\end{minipage}}

%%% languages
\usepackage{cmap}					% lookup in pdf
\usepackage{mathtext} 				% cyrillic letters in formulas
\usepackage[T2A]{fontenc}			% encoding
\usepackage[utf8]{inputenc}			% source encoding
\usepackage[english,russian]{babel}	% l18n and hyphenation

%%% mathematics
\usepackage{amsmath,amsfonts,amssymb,mathtools}

%%% перенос формул по Львовскому
\newcommand*{\hm}[1]{#1\nobreak\discretionary{}{\hbox{$\mathsurround=0pt #1$}}{}}

% totally disable breaking inline formulas across lines
\relpenalty=10000
\binoppenalty=10000

% reduce hyphenation
\lefthyphenmin=8
\righthyphenmin=8

%%% fonts
\usepackage{euscript,mathrsfs,pifont} % euclid font, cool math font, dingbats

%%% figures
\usepackage{graphicx,wrapfig,float}
\numberwithin{figure}{section}
\graphicspath{{images/}}
\newcommand\cfigure[2]{
	\begin{figure}[H] \centering \includegraphics[width=#1]{#2} \end{figure}
}
\newcommand\ccfigure[3]{
	\begin{figure}[H] \centering \includegraphics[width=#1]{#2} \caption{#3} \end{figure}
}

%%% links
\usepackage{color,hyperref}
\hypersetup{colorlinks=true, linkcolor=blue, filecolor=magenta, urlcolor=cyan}

%%% styles
\usepackage{amsthm}

\usepackage{parskip}      % space between paragraphs rather than indentation
\makeatletter             % fix \topsep in amsthm after \usepackage{parskip}
\def\thm@space@setup{\thm@preskip=2.5\parskip \thm@postskip=0pt}
\makeatother

\theoremstyle{definition}
\newtheorem{definition}{Опр-е.}[section]
\newtheorem*{property}{Св-во}   %[definition]
\newtheorem{theorem}{Tеор.}[section]
\newtheorem*{corollary}{След-е} %[theorem]
\newtheorem{lemma}{Лемма}[section]
\newtheorem{problem}{Задача}[section]

% reduce space between theorem and proof
\let\exproof\proof
\def\proof{\vspace{-.6em}\exproof[Док-во]}
\def\solution{\vspace{-.6em}\exproof[Решение]}
%\renewcommand\qedsymbol{$\blacksquare$}


%%% math mode macros
% russian-style greeks
\let\DS\displaystyle
\let\eps\varepsilon
\let\phi\varphi
\let\leqs\leqslant
\let\geqs\geqslant
\def\BB{\mathbb{B}}
\def\RR{\mathbb{R}}
\def\SS{\mathfrak{S}}
\def\mbyn{\dfrac{m}{n}}

%%% text mode macros
\def\lets{{\huge$\lrcorner$}\space}
\def\iff{$\;\Longleftrightarrow\;$}
\def\any{$\forall\;$}
\def\todo{\guillemotleft $\mathcal{TODO}$ \guillemotright\textellipsis}
\def\vignette{\vspace{48pt} \noindent \hrulefill~
	          \raisebox{-8pt}[10pt][10pt]{\Huge\ding{102}}
	          ~\hrulefill}


%%%%%%%%%%%%%%%%%%%%%%%%%%%%%%%%%%%%%%%%%%%%%%%%%%%%%%%%%%%%%%%%%%%%%%%
\title{Теория Вероятности}
\author{}
\date{}

\begin{document}

\maketitle
%\tableofcontents


%%%%%%%%%%%%%%%%%%%%%%%%%%%%%%%%%%%%%%%%%%%%%%%%%%%%%%%%%%%%%%%%%%%%%%%
\section{Непрерывные СВ}


%%%%%%%%%%%%%%%%%%%%%%%%%%%%%%%%%%%%%%%%%%%%%%%%%%%%%%%%%%%%%%%%%%%%%%%
\subsection{Вероятностное пространство}

\begin{definition}[$\sigma$-алгебра]
\textit{$\sigma$-алгебра над множеством $X$}
--- это семейство $\SS$ подмножеств множества $X$, т.ч.:
\begin{enumerate}
	\item $X \in \SS$ и $\varnothing \in \SS$
	\item если $E \in \SS$, то $X \setminus E \in \SS$
	\item если $\exists$ семейство $\{A_n\} \in \SS$ (конечное или счётное),
	      то $\DS \bigcup_{n=1}^{\infty} A_n \in \SS$
	      и $\DS \bigcap_{n=1}^{\infty} A_n \in \SS$
\end{enumerate}
\end{definition}

\begin{definition}[Борелевское множество]
	\textit{$\BB$-множество} --- такое множество, которое может быть
	получено из \textit{открытых или замкнутых промежутков} на $\RR$
	конечным или счетным числом операций $\bigcup A_n$ и $\bigcap A_n$.
\end{definition}

\begin{definition}
	\textit{Борелевская $\sigma$-алгебра} $\equiv$ минимальная
	$\sigma$-алгебра борелевских множеств на $\RR$.
\end{definition}


\medskip
\begin{problem}
	Является ли множество всех рациональных точек на прямой борелевским,
	т.е. верно ли, что
	$\left\{ \mbyn, n=1,2,...; m=0,1,2,... \right\} \in \BB$?
\end{problem}
\begin{solution}
	{\bfseries Да.}\\
	Каждую точку можно рассматривать как отрезок
	$\left[ \mbyn,\mbyn \right] =
	\left( -\infty, \mbyn \right] \setminus \left( -\infty, \mbyn \right)$,
	причём множество всех рациональных точек на прямой счетно.
\end{solution}


%%%%%%%%%%%%%%%%%%%%%%%%%%%%%%%%%%%%%%%%%%%%%%%%%%%%%%%%%%%%%%%%%%%%%%%
\subsection{Функции распределения и плотности}

Функция распределения вероятности (CDF)
\begin{enumerate}
	\item $F_X(x) = P(X \leqs x)$
	\item $0\leqs F_X \leqs 1$
	\item $F$ -- монотонно неубывающая
	\item $F(-\infty)=0$, $F(+\infty)=1$
	\item $P(a< X \leqs b) = F(b)-F(a)$
	\item $F$ -- непрерывна \textit{справа}
	\item $F$ -- это некоторая вероятность, т.е. безразмерная величина
\end{enumerate}

Замечание: если бы CDF была определена как
$P(X {\pmb{\textcolor{red}<}} x)$,
то была бы непрерывна \textit{слева}.

\cfigure{11cm}{cdf-example-big.png}

\cfigure{6cm}{cdf-example-small.jpg}

Функция плотности распределения (PDF):
\begin{enumerate}
	\item $f(u) \geqs 0$
	\item $f(x)=F'(x)=\dfrac{dF(x)}{dx}$
	\item $\DS P(X\leqs x)=F(x)=\int_{-\infty}^{x}f(u)du$
	\item $\DS \int_{-\infty}^{+\infty}f(u)du=1$ -- условие нормировки
	\item $P(X=a)=0$
	\item $P(X \in \left<x,x+\delta\right>) \approx f(x) \delta$
	\item размерность $f$ есть $\dfrac{1}{\text{размерность}(X)}$
		  (например, $\text{см}^{-1}$, $\text{кг}^{-1}$)
\end{enumerate}


%%%%%%%%%%%%%%%%%%%%%%%%%%%%%%%%%%%%%%%%%%%%%%%%%%%%%%%%%%%%%%%%%%%%%%%
\subsection{Характеристики непрерывных СВ}

\begin{onsamepage}
Матожидание $\DS M(x)=\int_{-\infty}^{+\infty}uf(u)du$

$M(x)$ существует \iff интеграл сходится \textit{абсолютно}:
$\DS \int_{-\infty}^{+\infty}|u|f(u)du < \infty$.
\end{onsamepage}

Матожидание может не существовать.
Пример -- распределение Коши: $f(x)=\dfrac{1/\pi}{1+x^2}$.
Для него $\DS \int_{-\infty}^{+\infty}|u|f(u)du
            = \frac2\pi \int_0^\infty \frac{x\,\mathrm{d}x}{1+x^2}
            = \frac1\pi \ln(1+x^2) \Big\rvert_0^\infty \to \infty$

Если $f(x)=f(-x)$ (чётная), то $M(X)=0$ (если существует).

\medskip

Матожидание функции
$\DS M\left( \phi(x) \right) = \int_{-\infty}^{+\infty}\phi(u)f(u)du$.

\medskip

Дисперсия $D(X) = M\left( X-M(X) \right) = M(X^2)-M^2(X)$

\bigskip

Медиана (характеристика \textit{положения}):
$\mathrm{Me}(X) = arg \left\{ F_X(x) = 1/2 \right\}$

Медиана существует \textit{всегда}.

Если $f(x)=f(-x)$ (чётная), то для нормировки функции плотности имеем:
\[ 1 = \int_{-\infty}^\infty f(u)du
     = \int_{-\infty}^0 f(u)du + \int_0^\infty f(u)du
     = F(0) + \int_\infty^0 f(-u)d(-u) = F(0) + F(0) \]
а значит $F(0)=1/2$, т.е. $\mathrm{Me}(X) = 0$.

\bigskip

Нижний квартиль: $F(Q_1) = 1/4$

Нижний квартиль: $F(Q_3) = 3/4$

Межквартильный размах: $IQR=Q_3-Q_1$

\medskip

Квантиль уровня $\alpha$: $F(q_{\alpha})=\alpha$

Квантильная функция $Q(p)=F^{-1}(x)$, $p\!\in\![0,1]$ --
функция, обратная к функции распределения


%%%%%%%%%%%%%%%%%%%%%%%%%%%%%%%%%%%%%%%%%%%%%%%%%%%%%%%%%%%%%%%%%%%%%%%
\subsection{Экспоненциальное распределение}

\cfigure{.9\linewidth}{exp-dist.png}

$X \sim Exp(\lambda)$

PDF: $\DS f(u) = \lambda e^{-\lambda u}$, $u\geqs0$, $\lambda>0$

CDF: $\DS F(X) = P(X\leqs x) = \int_0^x \lambda e^{-\lambda u} du = 1-e^{-\lambda x}$

\begin{align*}
	   f(u)&=\lambda e^{-\lambda u} \qquad u\geqs 0 ,\; \lambda>0
	\\ F(x)&=1-e^{-\lambda x} \qquad x\geqs 0
	\\ M(x)&=\frac{1}{\lambda}
	\\ D(x)&=\frac{1}{\lambda^2}
	\\ \mathrm{Me}(x)&=\frac{\ln 2}{\lambda} \qquad \text{скошено вправо: } Me<M
\end{align*}

Свойство отсутствия памяти:
\[ P(X>t+s|X>t) = \frac{P(X>t+s,X>t)}{P(X>t)} = \frac{P(X>t+s)}{P(X>t)} =
   \frac{e^{-\lambda(t+s)}}{e^{-\lambda t}}=P(X>s) \quad (s,t\!\geqs\!0) \]

Среди дискретных распределений свойством отсутствия памяти обладает геометрическое.
Экспоненциальное распределение -- непрерывный аналог геометрического.


%%%%%%%%%%%%%%%%%%%%%%%%%%%%%%%%%%%%%%%%%%%%%%%%%%%%%%%%%%%%%%%%%%%%%%%
\subsection{Процесс Пуассона}

\textit{Случайный процесс} -- последовательость случайных величин в дискретном времени
(обычно НОР, т.е. независимых одинаково распределённых).


Пуассоновский процесс -- непрерывный аналог процесса Бернулли:
	\[ P(k,t) = \frac{(\lambda t)^k}{k!} e^{-\lambda t} \]


Свойства ПП:
\begin{enumerate}
	\item стационарный (распределение числа событий зависит только от длины интервала)
	\item ординарный (вряд ли на очень малом интервале произойдёт больше одного события)
	\item поток с отсутствием памяти.
\end{enumerate}


\begin{problem}
	Будет ли пуассоновский процесс подходящей моделью процесса
	прибытия пассажиров в пункт выдачи багажа в аэропорту?
\end{problem}
\begin{solution}
  \textbf{Нет.}\\
	Не выполняется требование независимости событий, т.к.
	пассажиры приходят за багажом после посадки самолета и, если мы наблюдаем,
	к примеру, что в зону получения багажа за последнюю минуту пришли 15 человек,
	то следует ожидать значительное количество людей в течение следующей минуты.
	Как получают багаж? Сначала толпа, потом редкие одиночки.
\end{solution}


\begin{onsamepage}
\begin{problem}
	Будет ли пуассоновский процесс подходящей моделью процесса поступления
	звонков в регистратуру поликлиники в течение рабочего дня?
\end{problem}
\begin{solution}
  \textbf{Нет.}\\
	Не выполняется требование стационарности пуассоновского процесса, т.к.
	частота звонков намного выше в утренние часы в сравнении с вечерними.
	К концу дня поток звонков при любом распорядке будет снижаться
	(если это не ночной стационар).
\end{solution}
\end{onsamepage}


\subsection{Прочие распределения}

Равномерное распределение $X \sim Uniform(a,b)$:
\begin{align*}
	   F(x) &=
	   		\begin{cases}
	   			0	& x<a \\
		   		\dfrac{x-a}{b-a}	& a \leqs x \leqs b \\
		   		1	& x>b
			\end{cases}
	\\ f(x) &= \begin{cases}
				0		& x<a \\
				1/(b-a)	& a \leqs x \leqs b \\
				0		& x>b
			\end{cases}
	\\ Me = M &= (a+b)/2
	\\ D &= (b-a)^2/12
\end{align*}


Распределение Лапласа:
\begin{align*}
	   f &= \frac{\lambda}2 e^{-\lambda|x|}
	\\ M &= 0
	\\ D &= \frac{2}{\lambda^2}
\end{align*}


Распределение Коши:
\begin{align*}
	   f(x)=\dfrac{1/\pi}{1+x^2}
\end{align*}


\subsection{Вспомогательные формулы}

\begin{align*}
	   \int_0^\infty     e^{-ax}dx &= \frac{1}{a}
	\\ \int_0^\infty x   e^{-ax}dx &= \frac{1}{a^2}
	\\ \int_0^\infty x^2 e^{-ax}dx &= \frac{2}{a^3}
	\\ \int_0^\infty x^n e^{-ax}dx &= \frac{n!}{a^{n+1}}
\end{align*}

\begin{align*}
	   a^3-b^3 &= (a-b)(a^2+ab+b^2)
	\\ (a-b)^3 &= a^3-3a^2b+3ab^2-b^3
\end{align*}

%%%%%%%%%%%%%%%%%%%%%%%%%%%%%%%%%%%%%%%%%%%%%%%%%%%%%%%%%%%%%%%%%%%%%%%
%%%%%%%%%%%%%%%%%%%%%%%%%%%%%%%%%%%%%%%%%%%%%%%%%%%%%%%%%%%%%%%%%%%%%%%
\vignette

\end{document}
