% page margins
\documentclass[a4paper,12pt]{article}
\usepackage[margin=2.5cm]{geometry}
\setlength{\parindent}{0cm}

% languages
\usepackage{cmap}					% lookup in pdf
\usepackage{mathtext} 				% cyrillic letters in formulas
\usepackage[T2A]{fontenc}			% encoding
\usepackage[utf8]{inputenc}			% source encoding
\usepackage[english,russian]{babel}	% l18n and hyphenation

% mathematics
\usepackage{amsmath,amsfonts,amssymb,amsthm}

% fonts
\usepackage{euscript,mathrsfs,pifont} % euclid font, cool math font, dingbats

% tables
\usepackage{array,tabularx,tabulary,booktabs} % Дополнительная работа с таблицами
\usepackage{longtable,multirow}  % long tables, merge table rows

% figures
\usepackage{graphicx,wrapfig,float}
\numberwithin{figure}{section}
\graphicspath{{images/}}

% links
\usepackage{color,hyperref}
\hypersetup{
	colorlinks=true,
	linkcolor=blue,
	filecolor=magenta,      
	urlcolor=cyan,
}

% styles
\usepackage{mathtools,thmtools}
%\renewcommand\qedsymbol{$\blacksquare$}
\theoremstyle{definition}
\newtheorem{definition}{Опр-е.}[section]
\newtheorem*{property}{Св-во}  %[definition]
%\theoremstyle{plain}
%\theoremstyle{remark}
\theoremstyle{definition}
\newtheorem{theorem}{Tеор.}[section]
\newtheorem*{corollary}{След-е} %[theorem]
\newtheorem{lemma}{Лемма}[section]

\declaretheoremstyle[
spaceabove=0pt, spacebelow=24pt,
headfont=\normalfont\bfseries, headpunct={.},
notefont=\normalfont\mdseries, notebraces={(}{)},
%postheadspace=.5em,
postheadspace=\newline,
bodyfont=\normalfont, qed=\qedsymbol
]{solution}

\declaretheorem[name=Задача,numberwithin=section,style=definition]{problem}
\declaretheorem[name=Решение,numbered=no,style=solution]{solution}

% math mode macros
\def\.{\;\;}
\def\eps{\varepsilon}
\def\ntoinf{n \to \infty}
\def\ringU{\mathring{U}}
\def\mathN{\mathbb{N}}
\def\mathR{\mathbb{R}}
\def\leqs{\leqslant}
\def\geqs{\geqslant}

% text mode macros
\def\circleone{\ding{192}$\;$}
\def\circletwo{\ding{193}$\;$}
\def\lets {$\gimel\;$}
\def\iff {$\;\Longleftrightarrow\;$}
\def\any {$\forall\;$}
\def\rowak{$\displaystyle\sum_{k=1}^{\infty}a_k$}
\def\rowbk{$\displaystyle\sum_{k=1}^{\infty}b_k$}
\def\todo {\guillemotleft$\mathcal{TODO}$\guillemotright\textellipsis}
\def\vignette {\vspace{48pt} \noindent \hrulefill~ \raisebox{-8pt}[10pt][10pt]{\Huge\ding{102}}~ \hrulefill}

% figures

\newcommand\CenterFigure[2]{
	\begin{figure}[H]
		\centering
		\includegraphics[width=#1]{#2}
	\end{figure}
}

\newcommand\CenterFigureCaption[3]{
	\begin{figure}[H]
		\centering
		\includegraphics[width=#1]{#2}
		\caption{#3}
	\end{figure}
}


%%%%%%%%%%%%%%%%%%%%%%%%%%%%%%%%%%%%%%%%%%%%%%%%%%%%%%%%%%%%%%%%%%%%%%%
\title{Введение в математический анализ.}
\author{}
\date{}

\begin{document}

\maketitle
\tableofcontents


%%%%%%%%%%%%%%%%%%%%%%%%%%%%%%%%%%%%%%%%%%%%%%%%%%%%%%%%%%%%%%%%%%%%%%%
%%%%%%%%%%%%%%%%%%%%%%%%%%%%%%%%%%%%%%%%%%%%%%%%%%%%%%%%%%%%%%%%%%%%%%%
\vignette
\section{Последовательности}

\subsection{Предел последовательности}

\begin{definition}[Предел последовательности]
	\[ \lim_{\ntoinf} x_n = l \]
	\begin{itemize}
		\item При любом $\eps>0$ вне интервала $(l-\eps,l+\eps)$
			  находится лишь конечное число членов последовательности
		\item $\forall \eps>0 \quad \exists N \quad \forall n>N \quad |x_n - l|<\eps$
	\end{itemize}
\end{definition}

\begin{property}
	Свойства последовательностей:
	\begin{itemize}
	\item Не может иметь двух различных пределов
	\item Если имеет предел, то $|x_n| \leqs M$
	\item Переход к пределу в неравенстве: \[
		\lim_{\ntoinf}x_n=a, \lim_{\ntoinf}y_n=b, x_n \leqs y_n
		\implies a \leqs b
		\]
	\item Пределы можно складывать, вычитать, умножать, делить, брать модуль
	\end{itemize}
\end{property}

\begin{theorem}[О двух милиционерах]
	\[ a_n \leqs b_n \leqs c_n , \quad
	   \lim_{\ntoinf}a_n = \lim_{\ntoinf}b_n = A
	   \implies \lim_{\ntoinf}c_n = A \]
\end{theorem}


%%%%%%%%%%%%%%%%%%%%%%%%%%%%%%%%%%%%%%%%%%%%%%%%%%%%%
\subsection{Арифметические операции с пределами}

Арифметические операции с конечными пределами...

\begin{definition}[Бесконечный предел] \[
	\lim_{\ntoinf} x_n = +\infty
	\qquad\equiv\qquad
	\forall E \. \exists N \. \forall n>N \. x_n>E
	\]
\end{definition}

\begin{theorem}
	\lets $x_n \neq 0$. $x_n$ -- беск. большая \iff $\frac1{x_n}$ -- беск. малая
\end{theorem}

\begin{property}
	Свойства бесконечно малых:
	\begin{enumerate}
		\item Беск. малая послед. ограничена
		\item Сумма, разность, произведение бес. малых -- беск. малая
		\item Произвед. беск. малой на ограниченную -- беск. малая
	\end{enumerate}
\end{property}

Арифметические операции с бесконечностями...


%%%%%%%%%%%%%%%%%%%%%%%%%%%%%%%%%%%%%%%%%%%%%%%%%%%%%
\subsection{Вещественные числа. Супремум и инфимум.}

\begin{definition}[Вещественные числа] \ \\
	\begin{itemize}
		\item Аксиомы поля (9 штук)
		\item Аксимомы порядка (5 штук)
		\item Аксиома Архимеда:
				$ \forall\. x,y>0 \. \exists \; n \in \mathN \;: nx>y $
		\item Аксиома полноты:
			Пусть $[a_1,b_1] \supset [a_2,b_2] \supset [a_3,b_3] \supset...$.
			Тогда существует число $c \in \mathR$, принадлежащее всем отрезкам:
			$\displaystyle c \in \bigcap_{n=1}^{\infty} [a_n,b_n]$
	\end{itemize}
\end{definition}

\begin{theorem}[О стягивающихся отрезках] \ \\
	Пусть $[a_1,b_1] \supset [a_2,b_2] \supset [a_3,b_3] \supset ...$
	и $\displaystyle \lim_{\ntoinf}(b_n-a_n) = 0$.
	Тогда пересечение всех отрезков состоит из одной точки:
	$\displaystyle \{c\} = \bigcap_{n=1}^{\infty} [a_n,b_n]$, причём
	$\displaystyle \lim_{\ntoinf}a_n = \lim_{\ntoinf}b_n = c$
\end{theorem}

\begin{definition}
	\lets $E$ -- непустое множество \\
	$\sup$ -- наименьшая из верхних границ \\
	$\inf$ -- наибольшая из нижних границ
	\[ b = \sup E \Longleftrightarrow
		\begin{cases*}
			\forall x \in E \quad x \leqs b \\
			\forall \eps > 0 \. \exists\, x \in E : \;  x > b - \eps
		\end{cases*}
	\]
\end{definition}

\begin{theorem}
	Всякое непустое ограниченное сверху (снизу) множество имеет $\sup$ ($\inf$)
\end{theorem}
\begin{proof} Делением отрезка пополам... \end{proof}

\begin{theorem}  \ \\
	\begin{itemize}
		\item Монотонно возрастающая ограниченная сверху последовательность сходится.
		\item Монотонно убывающая ограниченная снизу последовательность сходится.
		\item Неограниченная сверху возрастающая последовательность стремится к $+\infty$.
		\item Неограниченная снизу убывающая последовательность стремится к $-\infty$.
	\end{itemize}
\end{theorem}


%%%%%%%%%%%%%%%%%%%%%%%%%%%%%%%%%%%%%%%%%%%%%%%%%%%%%
\subsection{Определение числа $e$}

\begin{lemma}[Неравенство Бернулли]
	\[ x>-1, n \in \mathN \implies (1+x)^n \geqs 1+nx \]
\end{lemma}
\begin{proof} По индукции... \end{proof}

\begin{corollary}
	\[ x>-1, n \in \mathN \implies \sqrt[n]{1+x} \leqs 1+\frac{x}{n} \]
\end{corollary}

\[ \lim_{\ntoinf} \frac{a^n}{n^k} = +\infty \quad a>1, \; k \in \mathN \]

\[ \lim_{\ntoinf} \sqrt[n]{a} = 1 \quad a>0 \]

\[ e = \lim_{\ntoinf}\left( 1 + \frac1n \right)^n \]


%%%%%%%%%%%%%%%%%%%%%%%%%%%%%%%%%%%%%%%%%%%%%%%%%%%%%
\subsection{Теорема Больцано-Вейерштрасса}

\begin{property}
	\[ \{ a_n \} \to A \implies \{ a_{n_k} \} \to A \]
\end{property}

\begin{theorem}[Больцано-Вейерштрасса]
	Из всякой ограниченной последовательности можно выделить сходящуюся (к конечному пределу) подпоследовательность
\end{theorem}

\begin{theorem}[Расширение теоремы Б-В]  \ \\
	\begin{itemize}
		\item Из неограниченной сверху последовательности можно выделить подпоследовательность, сходящуюся к $+\infty$.
		\item Из неограниченной снизу последовательности можно выделить подпоследовательность, сходящуюся к $-\infty$.
	\end{itemize}
\end{theorem}

\begin{corollary}
	Из любой последовательности можно выделить под-последовательность, имеющую конечный \textit{или бесконечный} предел.
\end{corollary}

\begin{definition}
	Последовательность фундаментальна, если
	$\forall \eps>0 \quad \exists N \quad \forall m,n \geqs N \quad |x_m - x_n|<\eps$
\end{definition}

\begin{property} \ \\
	\begin{enumerate}
		\item Фундаментальная последовательность ограничена
		\item Сходящаяся последовательность фундаментальна
		\item Если у фундаментальной последовательности есть сходящаяся подпоследовательность, то исходная последовательность сходится
	\end{enumerate}
\end{property}

\begin{corollary}[Критерий Коши]
	Последовательность сходится \iff она фундаментальна
\end{corollary}


%%%%%%%%%%%%%%%%%%%%%%%%%%%%%%%%%%%%%%%%%%%%%%%%%%%%%
\subsection{Сходимость рядов}

\begin{definition}
	\[ S_n = \sum_{k=1}^n a_k \]
	Если последовательность $\{S_n\} \to S$, то последовательность наз. сходящейся, а $S$ -- сумма ряда. Если $\{S_n\}$ не имеет предела или \textbf{бесконечный} предел, то ряд расходится.
\end{definition}

\begin{theorem}[Необходимое условие сходимости ряда] \ \\
	Если ряд $\displaystyle \sum_{k=1}^n a_k$ сходится,
	то $\displaystyle \lim_{\ntoinf}a_n = 0$.
\end{theorem}

Геометрическая прогрессия:
	\[ S_n = \sum_{k=1}^n aq^{k-1} = a\frac{1-q^n}{1-q} \]
	\[ \lim_{\ntoinf}S_n = \frac{a}{1-q} \quad \text{ при } |q|<1 \]

Гармонический ряд $\displaystyle H_n = 1 + \frac12 + \frac13 + ... + \frac1n$
расходится, т.к. $\displaystyle H_{2^n} \geqs \frac12$

\bigbreak
Пример:
\[ S_n = \sum_{k=1}^n \frac1{k(k+1)}
       = \frac1{1\cdot2} + \frac1{2\cdot3} + \frac1{3\cdot4} + ... + \frac1{n(n+1)}
       = 1 - \frac1{n+1} \]
\[ \lim_{\ntoinf}S_n = 1 \]

\begin{property} Свойства сходящихся рядов:
  \begin{enumerate}
	\item Ряд не может иметь двух различных сумм
	\item В сходящемся ряду можно произвольно расставлять скобки
		  (т.к. это будет подпоследовательность сходящейся последовательности)
	\item Добавление и отбрасывание конечного членов ряда не влияет на сходимость (но изменяет сумму)
	\item Сходящиеся ряды можно складывать и вычитать
	\item Сходящийся ряд можно домножать на константу
  \end{enumerate}
\end{property}


%%%%%%%%%%%%%%%%%%%%%%%%%%%%%%%%%%%%%%%%%%%%%%%%%%%%%
\subsection{Признаки сходимости рядов}

\begin{property}
	Если $a_k \geqs 0$, а последовательность $S_n$ ограничена сверху, то ряд сходится
\end{property}

\begin{property}[Признак сравнения]
	Если $0 \leqs a_k \leqs b_k$, то:
	\begin{itemize}
		\item если ряд \rowbk сходится, то ряд \rowak $\,$ сходится
		\item если ряд \rowak расходится, то ряд \rowbk $\,$ расходится
	\end{itemize}
\end{property}

Пример: ряд $\displaystyle\frac1{k^2}$ сходится. $\;$
$\displaystyle\sum_{k=1}^\infty \frac1{k^2} = \frac{\pi^2}6 $

\bigbreak
Пример: ряд $\displaystyle\frac1{\sqrt{k}}$ расходится.

\begin{theorem}[Признак Даламбера]
  Пусть $a_n > 0$. Тогда:
  \begin{enumerate}
	\item Если $\displaystyle\frac{a_{n+1}}{a_n} \leqs d < 1$, то ряд \rowak сходится
	\item Если $\displaystyle\frac{a_{n+1}}{a_n} \geqs 1$, то ряд расходится
	\item Пусть $\displaystyle d_* = \lim_{\ntoinf} \frac{a_{n+1}}{a_n}$. Тогда:
	  \begin{itemize}
	  	\item Если $d_* < 1$, то ряд сходится
	  	\item Если $d_* > 1$, то ряд расходится
	  	\item Если $d_* = 1$, то ряд может как сходиться, так и расходиться
	  \end{itemize}
  \end{enumerate}
\end{theorem}

\begin{theorem}[Признак Коши]
	Пусть $a_n>0$. Тогда:
	\begin{enumerate}
		\item Если $\displaystyle\sqrt[n]{a_n} \leqs d < 1$, то ряд \rowak сходится
		\item Если $\displaystyle\sqrt[n]{a_n} \geqs 1$, то ряд расходится
		\item Пусть $\displaystyle q_* = \lim_{\ntoinf} \sqrt[n]{a_n}$. Тогда:
		\begin{itemize}
			\item Если $d_* < 1$, то ряд сходится
			\item Если $d_* > 1$, то ряд расходится
			\item Если $d_* = 1$, то ряд может как сходиться, так и расходиться
		\end{itemize}
	\end{enumerate}
\end{theorem}

\begin{theorem}[Факт] \ \\
	Если $a_n>0$ и существует $\displaystyle \lim_{\ntoinf} \frac{a_{n+1}}{a_n}$,
	то также существует и $\displaystyle \lim_{\ntoinf} \sqrt[n]{a_n}$, и они равны.
\end{theorem}

\begin{theorem}[Признак Лейбница]
	Знакочередующийся ряд $a_1 - a_2 + a_3 - a_4 + ...$ с монотонно убывающим по абсолютной величине членом $a_1 \geqs a_2 \geqs a_3 \geqs ... > 0$
	сходится \iff $\displaystyle \lim_{\ntoinf} a_n = 0$
\end{theorem}

Пример -- ряд Лейбница: $\displaystyle 1-\frac13+\frac15-\frac17+\frac19-... = \frac\pi4$

\begin{definition}[Абсолютная сходимость] \ \\
	Ряд \rowak $\,$ сходится абсолютно, если ряд $\displaystyle\sum_{k=1}^{\infty}|a_k|$ сходится.
\end{definition}

\begin{theorem}
	Абсолютно сходящийся ряд сходится, причём
	$\displaystyle \left|\sum_{k=1}^{\infty}a_k\right| \leqs \sum_{k=1}^{\infty}|a_k|$
\end{theorem}
\begin{proof}
	Рассмотрим $0 \leqs a_k + |a_k| \leqs 2|a_k|$
\end{proof}

\bigbreak
Пример: если сходится ряд $\displaystyle \sum_{n=1}^{\infty}a_n^2$,
то сходится и ряд $\displaystyle \sum_{n=1}^{\infty}\frac{a_n}n$.

\bigbreak
Свойство: если ряд $\displaystyle \sum_{n=1}^{\infty}a_n$ сходится,
а ряд $\displaystyle \sum_{n=1}^{\infty}b_n$ расходится,
то ряд $\displaystyle \sum_{n=1}^{\infty} (a_n + b_n)$ расходится.


%%%%%%%%%%%%%%%%%%%%%%%%%%%%%%%%%%%%%%%%%%%%%%%%%%%%%
\subsection{Тесты на сходимость рядов}

\CenterFigure{\linewidth}{rows-test-1.png}

\CenterFigure{\linewidth}{rows-test-2.png}

\CenterFigure{\linewidth}{rows-test-3.png}

\CenterFigure{\linewidth}{rows-test-4.png}



%%%%%%%%%%%%%%%%%%%%%%%%%%%%%%%%%%%%%%%%%%%%%%%%%%%%%%%%%%%%%%%%%%%%%%%
%%%%%%%%%%%%%%%%%%%%%%%%%%%%%%%%%%%%%%%%%%%%%%%%%%%%%%%%%%%%%%%%%%%%%%%
\vignette
\section{Функции и непрерывность}


%%%%%%%%%%%%%%%%%%%%%%%%%%%%%%%%%%%%%%%%%%%%%%%%%%%%%
\subsection{Предел функции}

\subsubsection{Предельные точки множества}

\begin{definition} Окрестность точки $U_a$ -- любой интервал вида $(a-\eps,a+\eps)$ при $\eps>0$ \end{definition}

\begin{definition} Проколотая окрестность $\ringU_a = U_a \setminus \{a\}$ \end{definition}

\begin{definition} Окрестность $+\infty$ -- любой луч $(E,+\infty)$ \end{definition}

\begin{definition} Окрестность $-\infty$ -- любой луч $(-\infty,E)$ \end{definition}

\begin{definition}
	$a$ -- предельная точка множества $E \subset \mathR$,
	если $\ringU_a \cap E \neq \varnothing$ для любой $\ringU_a$
\end{definition}

\bigbreak
Примеры:\begin{enumerate}
	\item $[a,b]$ -- множество предельных точек $(a,b)$
	\item $\{a\}$ -- предельная точка ряда $\{a_n\}\xrightarrow[\ntoinf]{}a$
	\item $\varnothing$ -- нет предельных точек у одиночной точки $\{a\} \in \mathR$
\end{enumerate}

\begin{lemma}[Утверждение.]
  Следующие условия равносильны:
  \begin{enumerate}
	\item $a$ -- предельная точка множества $E$
	\item В \any окрестности точки $a$ найдётся бесконечно много точек из $E$
	\item $\exists$ такая последовательность точек $x_n \in E$ ($x_n \neq a$),
		  что $\displaystyle \lim_{\ntoinf}x_n=a$
  \end{enumerate}
\end{lemma}

\subsubsection{Предел функции}

\begin{definition}
	Пусть дана функция $f$: $E\to\mathR$, заданная на множестве $E \subset \mathR$.
	Пусть $a$ -- предельная точка множества $E$.
	Тогда $\displaystyle \lim_{x \to a}f(x)=A$ (или $f(x) \xrightarrow[x \to a]{} A$),
	если выполнено любое из равносильных условий:
	\begin{enumerate}
		\item Для \any окрестности $U_A$ $\exists$ такая окрестность $\ringU_a$,
			  что $f(\ringU_a \cap E) \subset U_A$
		\item $\forall \eps>0 \. \exists \delta>0 \. \forall x \in E$,
			  т.ч. $x \ne a$ $\implies$ $\left|f(x)-A\right|<\eps$
			  (определение по Коши)
		\item Для \any последовательности $\{x_n\}$ точек из $E$ ($x_n\ne a$),
			  т.ч. $\displaystyle \lim_{\ntoinf}x_n = a$
			  $\implies$ $\displaystyle \lim_{\ntoinf}f(x_n) = A$
			  (определение по Гейне)
	\end{enumerate}
\end{definition}

\bigbreak
Замечания к определению предела функции: \begin{enumerate}
	\item Предел -- локальное свойство
	\item Значение $f$ в точке $a$ не участвует в определении
	\item Если в определении по Гейне \any последовательность $f(x_n)$ имеет предел,
  		  то все эти пределы равны
\end{enumerate}

\bigbreak
Свойства пределов: \begin{enumerate}
	\item Предел единственный.
	\item Локальная ограниченность: если $f\!: E\to \mathR$, $a$ -- предельная точка $E$,
  		  $\displaystyle \lim_{x\to a}f(x)=A$ и $A \in \mathR$,
  		  то $\exists$ такая окрестность $U_a$, что $f(x)$ ограничена на $U_a \cap E$.
	\item Стабилизация знака: если $f\!: E\to \mathR$, $a$ -- предельная точка $E$,
		  $\displaystyle \lim_{x\to a}f(x)=A$ и $A \in \mathR \setminus \{0\}$,
  		  то $\exists$ такая окрестность $U_a$,
  		  что знаки $f(x)$ при $x \in \ringU_a \cap E$ и $A$ совпадают.
\end{enumerate}


\subsubsection{Арифметические действия с пределами}

Пределы двух функций в точке можно складывать, вычитать, перемножать и делить (если предел нижней функции не равен 0).


\begin{theorem}[Предельный переход в неравенстве]
  Если \begin{enumerate}
	\item $f,g :\, E\to\mathR$, $\;$ $a$ -- предельная точка $E$
	\item $f(x) \leqs g(x)$ при всех $x \in E \setminus \{a\}$
	\item $\displaystyle\lim_{x\to a}f(x)=A$, $\displaystyle\lim_{x\to a}g(x)=B$
  \end{enumerate}
  Тогда $A \leqs B$
\end{theorem}

\begin{theorem}[Теорема о сжатой функции (аналог теоремы о двух милиционерах)]
  Если \begin{enumerate}
	\item $f,g,h :\, E\to\mathR$, $\;$ $a$ -- предельная точка $E$
	\item $f(x) \leqs g(x) \leqs h(x)$ при всех $x \in E \setminus \{a\}$
	\item $\displaystyle \lim_{x\to a}f(x)=\lim_{x\to a}h(x)=A$
  \end{enumerate}
  Тогда $\displaystyle \lim_{x\to a}g(x)=A$
\end{theorem}

\subsubsection{Односторонние пределы}

\begin{definition}[Монотонная функция]
	$f :\, E\to\mathR$ монотонно возрастает (убывает), если для \any $x \leqs y$
	выполнено $f(x) \leqs f(y)$ (или $f(x) \geqs f(y)$)
\end{definition}


\begin{theorem}
	Пусть $f:\, E\to\mathR$, $\;$ $a$ -- предельная точка множества $E_1=E \cap (-\infty,a)$.
	Тогда: \begin{itemize}
		\item Если $f$ возрастает и ограничена сверху, то
			  $\exists$ $\displaystyle \lim_{x\to a-}f(x)$
		\item Если $f$ \textit{убывает} и ограничена \textit{снизу}, то
			  $\exists$ $\displaystyle \lim_{x\to a-}f(x)$
	\end{itemize}
\end{theorem}


%%%%%%%%%%%%%%%%%%%%%%%%%%%%%%%%%%%%%%%%%%%%%%%%%%%%%
\subsection{Непрерывность функции}

\begin{definition}
  $f\!: E\to\mathR$ непрерывна в точке $a \in E$,
  если выполнено любое из равносильных условий:
  \begin{enumerate}
	\item Если $a$ -- предельная точка, то $\displaystyle \lim_{x\to a}f(x)=f(a)$
	\item $\displaystyle \forall\eps>0 \. \exists\delta>0 \text{ т.ч. }
		   |x-a|<\delta \implies |f(x)-f(a)|<\eps$
	\item Для \any окрестности $U_{f(a)}$ $\exists$ такая окрестность $U_a$,
		  что $f(U_a \cap E) \subset U_{f(a)}$
	\item Для \any послед. точек $\{x_n\} \subset E$ т.ч.
		  $\displaystyle \lim_{\ntoinf}x_n=a \implies \lim_{\ntoinf}f(x_n)=f(a)$
  \end{enumerate}
\end{definition}

Непрерывные в точке $a$ функции можно складывать, вычитать, умножать и (если $g(a)\ne0$) делить.

\bigbreak
Следствия: \begin{enumerate}
	\item Многочлены $\displaystyle p(x)=\sum_{k=0}^n a_kx^k$ непрерывны на $\mathR$.
	\item Рациональные функции (отношения многочленов $\displaystyle \frac{p(x)}{q(x)}$)
		  непрерывны во всех точках, в которых знаменатель не обращается в ноль.
\end{enumerate}

\begin{theorem}[Теорема о стабилизации знака]
	Если $f\!: E\to\mathR$ непрерывна в точке $a \in E$ и $f(a)\ne0$,
	то найдётся такая окрестность $U_a$, что знак $f(x)$ совпадает с $f(a)$.
\end{theorem}

\begin{theorem}[Непрерывность композиции]
	Пусть $f\!: D\to\mathR, \. g\!: E\to\mathR \. f(D)\subset E$ и
	$f$ непрерывна в точке $a\in D$, а $g$ непрерывна в точке $f(a)$.
	Тогда $g \circ f$ непрерывна в точке $a$.
\end{theorem}

Вспомогательное нер-во: если $0<x<\frac\pi2$, то $\sin x<x<\tg x$.

%%%%%%%%%%%%%%%%%%%%%%%%%%%%%%%%%%%%%%%%%%%%%%%%%%%%%
\subsection{Теорема Вейерштрасса}

\begin{theorem}[Вейерштрасса]
  Непрерывная на \textit{отрезке} функция: \\
	\circleone ограничена;
	\circletwo принимает наибольшее и наименьшее значения
\end{theorem}

\bigbreak
Расширение теоремы: если функция $f$ непрерывна на $[a,+\infty]$,
и $\exists$ конечный предел $\displaystyle \lim_{x\to+\infty}f(x)$,
то $f$ ограничена на $[a,+\infty]$.


%%%%%%%%%%%%%%%%%%%%%%%%%%%%%%%%%%%%%%%%%%%%%%%%%%%%%
\subsection{Теорема Больцано-Коши}


%%%%%%%%%%%%%%%%%%%%%%%%%%%%%%%%%%%%%%%%%%%%%%%%%%%%%
\subsection{Замечательные пределы}


%%%%%%%%%%%%%%%%%%%%%%%%%%%%%%%%%%%%%%%%%%%%%%%%%%%%%
\subsection{Эквивалентные функции}




%%%%%%%%%%%%%%%%%%%%%%%%%%%%%%%%%%%%%%%%%%%%%%%%%%%%%%%%%%%%%%%%%%%%%%%
%%%%%%%%%%%%%%%%%%%%%%%%%%%%%%%%%%%%%%%%%%%%%%%%%%%%%%%%%%%%%%%%%%%%%%%
\vignette
\section{Производные}

%%%%%%%%%%%%%%%%%%%%%%%%%%%%%%%%%%%%%%%%%%%%%%%%%%%%%
\subsection{Дифференцируемость и производная}


%%%%%%%%%%%%%%%%%%%%%%%%%%%%%%%%%%%%%%%%%%%%%%%%%%%%%
\subsection{Теоремы о среднем}

\todo


%%%%%%%%%%%%%%%%%%%%%%%%%%%%%%%%%%%%%%%%%%%%%%%%%%%%%
\subsection{Производная и монотонность}


%%%%%%%%%%%%%%%%%%%%%%%%%%%%%%%%%%%%%%%%%%%%%%%%%%%%%
\subsection{Правило Лопиталя}


%%%%%%%%%%%%%%%%%%%%%%%%%%%%%%%%%%%%%%%%%%%%%%%%%%%%%
\subsection{Формула Тейлора}


%%%%%%%%%%%%%%%%%%%%%%%%%%%%%%%%%%%%%%%%%%%%%%%%%%%%%
\subsection{Экстремумы функций}




%%%%%%%%%%%%%%%%%%%%%%%%%%%%%%%%%%%%%%%%%%%%%%%%%%%%%%%%%%%%%%%%%%%%%%%
%%%%%%%%%%%%%%%%%%%%%%%%%%%%%%%%%%%%%%%%%%%%%%%%%%%%%%%%%%%%%%%%%%%%%%%
\vignette
\section{Интегралы}


%%%%%%%%%%%%%%%%%%%%%%%%%%%%%%%%%%%%%%%%%%%%%%%%%%%%%
\subsection{Первообразная и неопределённый интеграл}

\todo


%%%%%%%%%%%%%%%%%%%%%%%%%%%%%%%%%%%%%%%%%%%%%%%%%%%%%
\subsection{Действия с неопределёнными интегралами}


%%%%%%%%%%%%%%%%%%%%%%%%%%%%%%%%%%%%%%%%%%%%%%%%%%%%%
\subsection{Площади и определённый интеграл}


%%%%%%%%%%%%%%%%%%%%%%%%%%%%%%%%%%%%%%%%%%%%%%%%%%%%%
\subsection{Теорема Барроу и формула Ньютона-Лейбница}


%%%%%%%%%%%%%%%%%%%%%%%%%%%%%%%%%%%%%%%%%%%%%%%%%%%%%
\subsection{Интегральные суммы}


%%%%%%%%%%%%%%%%%%%%%%%%%%%%%%%%%%%%%%%%%%%%%%%%%%%%%
\subsection{Связь между суммами и интегралами}

\vignette

\end{document}
