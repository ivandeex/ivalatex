% page margins
\documentclass[a4paper,12pt]{article}
\usepackage[margin=2.5cm]{geometry}
\setlength{\parindent}{0cm}  % disable indentation

% languages
\usepackage{cmap}					% lookup in pdf
\usepackage{mathtext} 				% cyrillic letters in formulas
\usepackage[T2A]{fontenc}			% encoding
\usepackage[utf8]{inputenc}			% source encoding
\usepackage[english,russian]{babel}	% l18n and hyphenation

% mathematics
\usepackage{amsmath,amsfonts,amssymb,amsthm}

% перенос формул по Львовскому
\newcommand*{\hm}[1]{#1\nobreak\discretionary{}{\hbox{$\mathsurround=0pt #1$}}{}}

% fonts
\usepackage{euscript,mathrsfs,pifont} % euclid font, cool math font, dingbats

% tables
\usepackage{array,tabularx,tabulary,booktabs} % Дополнительная работа с таблицами
\usepackage{longtable,multirow}  % long tables, merge table rows

% figures
\usepackage{graphicx,wrapfig,float}
\numberwithin{figure}{section}
\graphicspath{{images/}}

% links
\usepackage{color,hyperref}
\hypersetup{
	colorlinks=true,
	linkcolor=blue,
	filecolor=magenta,      
	urlcolor=cyan,
}

% styles
\usepackage{mathtools,thmtools}
%\renewcommand\qedsymbol{$\blacksquare$}
\theoremstyle{definition}
\newtheorem{definition}{Опр-е.}[section]
\newtheorem*{property}{Св-во}  %[definition]
%\theoremstyle{plain}
%\theoremstyle{remark}
\theoremstyle{definition}
\newtheorem{theorem}{Tеор.}[section]
\newtheorem*{corollary}{След-е} %[theorem]
\newtheorem{lemma}{Лемма}[section]

\declaretheoremstyle[
spaceabove=0pt, spacebelow=24pt,
headfont=\normalfont\bfseries, headpunct={.},
notefont=\normalfont\mdseries, notebraces={(}{)},
%postheadspace=.5em,
postheadspace=\newline,
bodyfont=\normalfont, qed=\qedsymbol
]{solution}

\declaretheorem[name=Задача,numberwithin=section,style=definition]{problem}
\declaretheorem[name=Решение,numbered=no,style=solution]{solution}

% math mode macros
\def\DS{\displaystyle}
\def\NN{\mathbb{N}}
\def\RR{\mathbb{R}}
\def\.{\;\;}
\def\on{\!:}
\def\eps{\varepsilon}
\def\ntoinf{n\to\infty}
\def\ringU{\mathring{U}}
\def\leqs{\leqslant}
\def\geqs{\geqslant}
\def\intab{\left<a,b\right>}

% text mode macros
\def\circleone{\ding{192}\space}
\def\circletwo{\ding{193}\space}
\def\circlethree{\ding{194}\space}
\def\lets{{\huge$\lrcorner$}\space}
\def\iff{$\;\Longleftrightarrow\;$}
\def\any{$\forall\;$}
\def\rowak{$\DS\sum_{k=1}^{\infty}a_k$\space}
\def\todo{\guillemotleft$\mathcal{TODO}$\guillemotright\textellipsis}
\def\vignette{\vspace{48pt} \noindent \hrulefill~ \raisebox{-8pt}[10pt][10pt]{\Huge\ding{102}}~ \hrulefill}

% figures

\newcommand\CenterFigure[2]{
	\begin{figure}[H]
		\centering
		\includegraphics[width=#1]{#2}
	\end{figure}
}

\newcommand\CenterFigureCaption[3]{
	\begin{figure}[H]
		\centering
		\includegraphics[width=#1]{#2}
		\caption{#3}
	\end{figure}
}


%%%%%%%%%%%%%%%%%%%%%%%%%%%%%%%%%%%%%%%%%%%%%%%%%%%%%%%%%%%%%%%%%%%%%%%
\title{Введение в математический анализ.}
\author{}
\date{}

\begin{document}

\maketitle
\tableofcontents


%%%%%%%%%%%%%%%%%%%%%%%%%%%%%%%%%%%%%%%%%%%%%%%%%%%%%%%%%%%%%%%%%%%%%%%
%%%%%%%%%%%%%%%%%%%%%%%%%%%%%%%%%%%%%%%%%%%%%%%%%%%%%%%%%%%%%%%%%%%%%%%
\vignette
\section{Последовательности}

\subsection{Предел последовательности}

\begin{definition}[Предел последовательности]
	$\DS \lim_{\ntoinf} x_n = a $
	\begin{itemize}
		\item При любом $\eps>0$ вне интервала $(a-\eps,a+\eps)$
			  находится лишь конечное число членов последовательности
		\item $\forall \eps>0 \. \exists N \. \forall n>N \. |x_n - a|<\eps$
	\end{itemize}
\end{definition}
\bigskip

Свойства последовательностей:
\begin{itemize}
	\item Не может иметь двух различных пределов
	\item Если имеет предел, то $|x_n| \leqs M$
	\item Пределы можно складывать, вычитать, умножать, делить, брать модуль
\end{itemize}
\bigskip

\begin{property}[Переход к пределу в неравенстве]
\[ \lim_{\ntoinf}x_n=a ,\; \lim_{\ntoinf}y_n=b ,\; x_n \leqs y_n
   \implies a \leqs b \]
\end{property}
\begin{proof} От противного \end{proof}
\bigskip

\begin{theorem}[О двух милиционерах]
	\[ a_n \leqs c_n \leqs b_n , \quad
	   \lim_{\ntoinf}a_n = \lim_{\ntoinf}b_n = A
	   \implies \lim_{\ntoinf}c_n = A \]
\end{theorem}
\begin{proof} По определению \end{proof}


%%%%%%%%%%%%%%%%%%%%%%%%%%%%%%%%%%%%%%%%%%%%%%%%%%%%%
\subsection{Арифметические операции с пределами}

\begin{property}
	Конечные пределы можно складывать, вычитать, умножать,
	делить (нижний д.б. $\ne0$) и брать модуль.
\end{property}
\begin{proof} С умом подбираем $\eps$ и используем ограниченность... \end{proof}
\bigskip

\begin{definition}[Бесконечный предел] \[
	\lim_{\ntoinf} x_n = +\infty
	\qquad\equiv\qquad
	\forall E \. \exists N \. \forall n>N \. x_n>E
\]
\end{definition}

\begin{theorem}
	\lets $x_n \neq 0$. $x_n$ -- беск. большая \iff $\DS\frac1{x_n}$ -- беск. малая
\end{theorem}

\begin{property}
	Свойства бесконечно малых:
	\begin{enumerate}
		\item Беск. малая послед. ограничена
		\item Сумма, разность, произведение бес. малых -- беск. малая
		\item Произвед. беск. малой на ограниченную -- беск. малая
	\end{enumerate}
\end{property}
\bigskip

Арифметические операции с бесконечностями...


%%%%%%%%%%%%%%%%%%%%%%%%%%%%%%%%%%%%%%%%%%%%%%%%%%%%%
\subsection{Вещественные числа. Супремум и инфимум.}

\begin{definition}[Вещественные числа] \ \\
	\begin{itemize}
		\item Аксиомы поля (9 штук)
		\item Аксимомы порядка (5 штук)
		\item Аксиома Архимеда:
				$ \forall\. x,y>0 \. \exists \; n \in \NN \;: nx>y $
		\item Аксиома полноты:
			\lets $[a_1,b_1] \supset [a_2,b_2] \supset [a_3,b_3] \supset...$.
			Тогда существует число $c \in \RR$, принадлежащее всем отрезкам:
			$\DS c \in \bigcap_{n=1}^{\infty} [a_n,b_n]$
	\end{itemize}
\end{definition}
\bigskip

\begin{theorem}[О стягивающихся отрезках] \ \\
	\lets $[a_1,b_1] \supset [a_2,b_2] \supset [a_3,b_3] \supset ...$
	и $\DS \lim_{\ntoinf}(b_n-a_n) = 0$.
	Тогда пересечение всех отрезков состоит из одной точки:
	$\DS \{c\} = \bigcap_{n=1}^{\infty} [a_n,b_n]$, причём
	$\DS \lim_{\ntoinf}a_n = \lim_{\ntoinf}b_n = c$
\end{theorem}
\begin{proof} По теореме о двух милиционерах \end{proof}
\bigskip

\begin{definition}
	\lets $E$ -- непустое множество \\
	$\sup$ -- наименьшая из верхних границ \\
	$\inf$ -- наибольшая из нижних границ
	\[ b = \sup E \Longleftrightarrow
		\begin{cases*}
			\forall x \in E \quad x \leqs b \\
			\forall \eps > 0 \. \exists\, x \in E : \;  x > b - \eps
		\end{cases*}
	\]
\end{definition}
\bigskip

\begin{theorem}
	Всякое непустое ограниченное сверху (снизу) множество имеет $\sup$ ($\inf$)
\end{theorem}
\begin{proof} Делением отрезка пополам, затем переход к пределу в неравенстве... \end{proof}
\bigskip

\begin{theorem}  \ \\
	\begin{itemize}
		\item Монотонно возрастающая ограниченная сверху последовательность сходится.
		\item Монотонно убывающая ограниченная снизу последовательность сходится.
	\end{itemize}
\end{theorem}
\begin{proof} Супремум... \end{proof}
\bigskip

\begin{theorem}  \ \\
	\begin{itemize}
		\item Неограниченная сверху возрастающая последовательность стремится к $+\infty$.
		\item Неограниченная снизу убывающая последовательность стремится к $-\infty$.
	\end{itemize}
\end{theorem}
\begin{proof} По определению \end{proof}
\bigskip


%%%%%%%%%%%%%%%%%%%%%%%%%%%%%%%%%%%%%%%%%%%%%%%%%%%%%
\subsection{Определение числа $e$}

\begin{lemma}[Неравенство Бернулли]
	\[ x>-1, n \in \NN \implies (1+x)^n \geqs 1+nx \]
\end{lemma}
\begin{proof} По индукции \end{proof}

\begin{corollary}
	\[ x>-1, n \in \NN \implies \sqrt[n]{1+x} \leqs 1+\frac{x}{n} \]
\end{corollary}

\[ \lim_{\ntoinf} \frac{a^n}{n^k} = +\infty \quad a>1, \; k \in \NN \]

\[ \lim_{\ntoinf} \sqrt[n]{a} = 1 \quad a>0 \]

\[ e = \lim_{\ntoinf}\left( 1 + \frac1n \right)^n \]


%%%%%%%%%%%%%%%%%%%%%%%%%%%%%%%%%%%%%%%%%%%%%%%%%%%%%
\subsection{Теорема Больцано-Вейерштрасса}

\begin{property}
	\[ \{ a_n \} \to A \implies \{ a_{n_k} \} \to A \]
\end{property}
\begin{proof} По определению \end{proof}
\bigskip

\begin{theorem}[Больцано-Вейерштрасса]
	Из всякой ограниченной последовательности можно выделить сходящуюся (к конечному пределу) подпоследовательность
\end{theorem}
\begin{proof}
	Делим содержащий последовательность отрезок пополам так,
	чтобы в нём оставалась бесконечная подпоследовательность.
	Далее по теореме о двух милиционерах.
\end{proof}
\bigskip

\begin{theorem}[Расширение теоремы Больцано-Вейерштрасса]  \ \\
	\begin{itemize}
		\item Из неограниченной сверху последовательности
			  можно выделить подпоследовательность, сходящуюся к $+\infty$.
		\item Из неограниченной снизу последовательности
			  можно выделить подпоследовательность, сходящуюся к $-\infty$.
	\end{itemize}
\end{theorem}
\bigskip

\begin{corollary}
	Из любой последовательности можно выделить под-последовательность,
	имеющую конечный \textit{или бесконечный} предел.
\end{corollary}
\bigskip


\begin{definition}
	Последовательность фундаментальна, если
	\[ \forall \eps>0 \. \exists N :\. \forall m,n \geqs N \quad |x_m - x_n|<\eps \]
\end{definition}

\medskip
Свойства: \begin{enumerate}
	\item Фундаментальная последовательность ограничена
	\item Сходящаяся последовательность фундаментальна
	\item Если у фундаментальной последовательности есть сходящаяся
		  подпоследовательность, то исходная последовательность сходится
\end{enumerate}
\bigskip

\begin{corollary}[Критерий Коши]
	Последовательность сходится \iff она фундаментальна
\end{corollary}


%%%%%%%%%%%%%%%%%%%%%%%%%%%%%%%%%%%%%%%%%%%%%%%%%%%%%
\subsection{Сходимость рядов}

\begin{definition}
	\[ S_n = \sum_{k=1}^n a_k \]
	Если последовательность $\{S_n\} \to S$,
	то последовательность наз. сходящейся, а $S$ -- сумма ряда.
	Если $\{S_n\}$ не имеет предела или \textit{бесконечный} предел,
	то ряд расходится.
\end{definition}
\bigskip

\begin{theorem}[Необходимое условие сходимости ряда] \ \\
	Если ряд $\DS \sum_{k=1}^n a_k$ сходится,
	то $\DS \lim_{\ntoinf}a_n = 0$.
\end{theorem}
\bigskip

Геометрическая прогрессия:
	\[ S_n = \sum_{k=1}^n aq^{k-1} = a\frac{1-q^n}{1-q} \]
	\[ \lim_{\ntoinf}S_n = \frac{a}{1-q} \quad \text{ при } |q|<1 \]

Гармонический ряд $\DS H_n = 1 + \frac12 + \frac13 + ... + \frac1n$
расходится, т.к. $\DS H_{2^n} \geqs \frac12$

\bigbreak
Пример:
\[ S_n = \sum_{k=1}^n \frac1{k(k+1)}
       = \frac1{1\cdot2} + \frac1{2\cdot3} + \frac1{3\cdot4} + ... + \frac1{n(n+1)}
       = 1 - \frac1{n+1} \]
\[ \lim_{\ntoinf}S_n = 1 \]
\bigskip

Свойства сходящихся рядов:
\begin{enumerate}
	\item Ряд не может иметь двух различных сумм
	\item В сходящемся ряду можно произвольно расставлять скобки
		  (т.к. это будет подпоследовательность сходящейся последовательности)
	\item Добавление и отбрасывание конечного членов ряда не влияет на сходимость (но изменяет сумму)
	\item Сходящиеся ряды можно складывать и вычитать
	\item Сходящийся ряд можно домножать на константу
\end{enumerate}


%%%%%%%%%%%%%%%%%%%%%%%%%%%%%%%%%%%%%%%%%%%%%%%%%%%%%
\subsection{Признаки сходимости рядов}

\begin{property}
	Если $a_k\geqs0$, а последовательность $S_n$ ограничена сверху, то ряд сходится
\end{property}

\begin{property}[Признак сравнения]
	Если $0 \leqs a_k \leqs b_k$, то:
	\begin{itemize}
		\def\rowbk{$\DS\sum_{k=1}^{\infty}b_k$\space}
		\item если ряд \rowbk сходится, то ряд \rowak сходится
		\item если ряд \rowak расходится, то ряд \rowbk расходится
	\end{itemize}
\end{property}

Пример: ряд $\DS\frac1{k^2}$ сходится. $\;$
$\DS\sum_{k=1}^\infty \frac1{k^2} = \frac{\pi^2}6 $

\bigbreak
Пример: ряд $\DS\frac1{\sqrt{k}}$ расходится.

\begin{theorem}[Признак Даламбера]
  \lets $a_n > 0$. Тогда:
  \begin{enumerate}
	\item Если $\DS\frac{a_{n+1}}{a_n} \leqs d < 1$, то ряд \rowak сходится
	\item Если $\DS\frac{a_{n+1}}{a_n} \geqs 1$, то ряд расходится
	\item \lets $\DS d_* = \lim_{\ntoinf} \frac{a_{n+1}}{a_n}$. Тогда:
	  \begin{itemize}
	  	\item Если $d_* < 1$, то ряд сходится
	  	\item Если $d_* > 1$, то ряд расходится
	  	\item Если $d_* = 1$, то ряд может как сходиться, так и расходиться
	  \end{itemize}
  \end{enumerate}
\end{theorem}

\begin{theorem}[Признак Коши]
	\lets $a_n>0$. Тогда:
	\begin{enumerate}
		\item Если $\DS\sqrt[n]{a_n} \leqs d < 1$, то ряд \rowak сходится
		\item Если $\DS\sqrt[n]{a_n} \geqs 1$, то ряд расходится
		\item \lets $\DS q_* = \lim_{\ntoinf} \sqrt[n]{a_n}$. Тогда:
		\begin{itemize}
			\item Если $d_* < 1$, то ряд сходится
			\item Если $d_* > 1$, то ряд расходится
			\item Если $d_* = 1$, то ряд может как сходиться, так и расходиться
		\end{itemize}
	\end{enumerate}
\end{theorem}

\begin{theorem}[Факт] \ \\
	Если $a_n>0$ и существует $\DS \lim_{\ntoinf} \frac{a_{n+1}}{a_n}$,
	то также существует и $\DS \lim_{\ntoinf} \sqrt[n]{a_n}$, и они равны.
\end{theorem}
\bigskip

\begin{theorem}[Признак Лейбница]
	Знакочередующийся ряд $a_1 - a_2 + a_3 - a_4 + ...$
	с монотонно убывающим по абсолютной величине членом
	$a_1 \geqs a_2 \geqs a_3 \geqs ... > 0$
	сходится \iff $\DS \lim_{\ntoinf} a_n = 0$
\end{theorem}
\begin{proof}
	Построить последовательность вложенных отрезков из частичных сумм
	$S_{2n-1}$, $S_{2n}$, $S_{2n+1}$, $S_{2n+2}$
\end{proof}
\bigskip

Пример -- ряд Лейбница: $\DS 1-\frac13+\frac15-\frac17+\frac19-... = \frac\pi4$
\bigskip


\begin{definition}[Абсолютная сходимость] \ \\
	Ряд \rowak сходится абсолютно,
	если ряд $\DS\sum_{k=1}^{\infty}|a_k|$ сходится.
\end{definition}

\begin{theorem}
	Абсолютно сходящийся ряд сходится, причём
	$\DS \left|\sum_{k=1}^{\infty}a_k\right| \leqs \sum_{k=1}^{\infty}|a_k|$
\end{theorem}
\begin{proof} Рассмотреть $0 \leqs a_k + |a_k| \leqs 2|a_k|$ \end{proof}

\bigbreak
Пример: если сходится ряд $\DS \sum_{n=1}^{\infty}a_n^2$,
то сходится и ряд $\DS \sum_{n=1}^{\infty}\frac{a_n}n$.
\\ \Big(доказательство: из $\sqrt{ab}\leqs\frac12(a+b)$ \Big)

\bigbreak
Свойство: если ряд $\DS \sum_{n=1}^{\infty}a_n$ сходится,
а ряд $\DS \sum_{n=1}^{\infty}b_n$ расходится,
то ряд $\DS \sum_{n=1}^{\infty} (a_n + b_n)$ расходится
(доказательство: \textit{от противного}).


%%%%%%%%%%%%%%%%%%%%%%%%%%%%%%%%%%%%%%%%%%%%%%%%%%%%%
\subsection{Тесты на сходимость рядов}

\CenterFigure{\linewidth}{rows-test-1.png}

\CenterFigure{\linewidth}{rows-test-2.png}

\CenterFigure{\linewidth}{rows-test-3.png}

\CenterFigure{\linewidth}{rows-test-4.png}



%%%%%%%%%%%%%%%%%%%%%%%%%%%%%%%%%%%%%%%%%%%%%%%%%%%%%%%%%%%%%%%%%%%%%%%
%%%%%%%%%%%%%%%%%%%%%%%%%%%%%%%%%%%%%%%%%%%%%%%%%%%%%%%%%%%%%%%%%%%%%%%
\vignette
\section{Функции и непрерывность}


%%%%%%%%%%%%%%%%%%%%%%%%%%%%%%%%%%%%%%%%%%%%%%%%%%%%%
\subsection{Предел функции}

\subsubsection{Предельные точки множества}

\begin{definition} Окрестность точки $U_a$ -- любой интервал вида $(a-\eps,a+\eps)$ при $\eps>0$ \end{definition}

\begin{definition} Проколотая окрестность $\ringU_a = U_a \setminus \{a\}$ \end{definition}

\begin{definition} Окрестность $+\infty$ -- любой луч $(E,+\infty)$ \end{definition}

\begin{definition} Окрестность $-\infty$ -- любой луч $(-\infty,E)$ \end{definition}
\bigskip

\begin{definition}
	$a$ -- предельная точка множества $E \subset \RR$,
	если $\ringU_a \cap E \neq \varnothing$ для любой $\ringU_a$
\end{definition}

\bigbreak
Примеры:\begin{enumerate}
	\item $[a,b]$ -- множество предельных точек $(a,b)$
	\item $\{a\}$ -- предельная точка ряда $\{a_n\}\xrightarrow[\ntoinf]{}a$
	\item $\varnothing$ -- нет предельных точек у одиночной точки $\{a\} \in \RR$
\end{enumerate}
\bigskip

\begin{lemma}[Утверждение.]
  Следующие условия равносильны:
  \begin{enumerate}
	\item $a$ -- предельная точка множества $E$
	\item В \any окрестности точки $a$ найдётся бесконечно много точек из $E$
	\item $\exists$ такая последовательность точек $x_n \in E$ ($x_n \neq a$),
		  что $\DS \lim_{\ntoinf}x_n=a$
  \end{enumerate}
\end{lemma}

\subsubsection{Предел функции}

\begin{definition}
	\lets дана функция $f$: $E\to\RR$, заданная на множестве $E \subset \RR$.
	\lets $a$ -- предельная точка множества $E$.
	Тогда $\DS \lim_{x \to a}f(x)=A$
	\big(или $f(x) \xrightarrow[x \to a]{} A$ \big),
	если выполнено любое из равносильных условий:
	\begin{enumerate}
		\item Для \any окрестности $U_A$ $\exists$ такая окрестность $\ringU_a$,
			  что $f(\ringU_a \cap E) \subset U_A$
		\item $\forall \eps>0 \. \exists \delta>0 \. \forall x \in E$,
			  т.ч. $x \ne a$ и $|x-a|<\delta$
			  $\implies$ $\left|f(x)-A\right|<\eps$
			  (определение по Коши)
		\item Для \any последовательности $\{x_n\}$ точек из $E$ ($x_n\ne a$),
			  т.ч. $\DS \lim_{\ntoinf}x_n = a$
			  $\implies$ $\DS \lim_{\ntoinf}f(x_n) = A$
			  (определение по Гейне)
	\end{enumerate}
\end{definition}
\bigskip

Замечания к определению предела функции: \begin{enumerate}
	\item Предел -- локальное свойство
	\item Значение $f$ в точке $a$ не участвует в определении
	\item Если в определении по Гейне \any последовательность $f(x_n)$ имеет предел,
  		  то все эти пределы равны (иначе мы могли бы построить последовательность
  		  перемешиванием, и она не имела бы предела)
  	\item В определении по Гейне достаточно ограничиться \textit{только}
  		  последовательностями, которые \textit{монотонно} стремятся к $a$
\end{enumerate}
\bigskip

Свойства пределов: \begin{enumerate}
	\item Предел единственный.
	\item Локальная ограниченность: если $f\on E\to\RR$, $a$ -- предельная точка $E$,
  		  $\DS \lim_{x\to a}f(x)=A$ и $A \in \RR$,
  		  то $\exists$ такая окрестность $U_a$, что $f(x)$ ограничена на $U_a \cap E$.
	\item Стабилизация знака: если $f\on E\to\RR$, $a$ -- предельная точка $E$,
		  $\DS \lim_{x\to a}f(x)=A$ и $A \in \RR \setminus \{0\}$,
  		  то $\exists$ такая окрестность $U_a$,
  		  что знаки $f(x)$ при $x \in \ringU_a \cap E$ и $A$ совпадают.
\end{enumerate}


\subsubsection{Арифметические действия с пределами}

Пределы двух функций в точке можно складывать, вычитать, перемножать и делить (если предел нижней функции не равен 0).
\bigskip

\begin{theorem}[Предельный переход в неравенстве]
  Если \begin{enumerate}
	\item $f,g :\, E\to\RR$, $\;$ $a$ -- предельная точка $E$
	\item $f(x) \leqs g(x)$ при всех $x \in E \setminus \{a\}$
	\item $\DS\lim_{x\to a}f(x)=A$, $\DS\lim_{x\to a}g(x)=B$
  \end{enumerate}
  Тогда $A \leqs B$
\end{theorem}

\begin{theorem}[Теорема о сжатой функции (аналог теоремы о двух милиционерах)]
  Если \begin{enumerate}
	\item $f,g,h :\, E\to\RR$, $\;$ $a$ -- предельная точка $E$
	\item $f(x) \leqs g(x) \leqs h(x)$ при всех $x \in E \setminus \{a\}$
	\item $\DS \lim_{x\to a}f(x)=\lim_{x\to a}h(x)=A$
  \end{enumerate}
  Тогда $\DS \lim_{x\to a}g(x)=A$
\end{theorem}

\subsubsection{Односторонние пределы}

\begin{definition}[Монотонная функция]
	$f :\, E\to\RR$ монотонно возрастает (убывает), если для \any $x \leqs y$
	выполнено $f(x) \leqs f(y)$ \big(или $f(x) \geqs f(y)$\big)
\end{definition}


\begin{theorem}
	\lets $f:\, E\to\RR$, $\;$ $a$ -- предельная точка множества $E_1=E \cap (-\infty,a)$.
	Тогда: \begin{itemize}
		\item Если $f$ возрастает и ограничена сверху, то
			  $\exists$ $\DS \lim_{x\to a-}f(x)$
		\item Если $f$ \textit{убывает} и ограничена \textit{снизу}, то
			  $\exists$ $\DS \lim_{x\to a-}f(x)$
	\end{itemize}
\end{theorem}


%%%%%%%%%%%%%%%%%%%%%%%%%%%%%%%%%%%%%%%%%%%%%%%%%%%%%
\subsection{Непрерывность функции}

\begin{definition}
  $f\on E\to\RR$ непрерывна в точке $a \in E$,
  если выполнено любое из равносильных условий:
  \begin{enumerate}
	\item Если $a$ -- предельная точка, то $\DS \lim_{x\to a}f(x)=f(a)$
	\item $\DS \forall\eps>0 \. \exists\delta>0 \text{ т.ч. }
		   |x-a|<\delta \implies |f(x)-f(a)|<\eps$
	\item Для \any окрестности $U_{f(a)}$ $\exists$ такая окрестность $U_a$,
		  что $f(U_a \cap E) \subset U_{f(a)}$
	\item Для \any послед. точек $\{x_n\} \subset E$ т.ч.
		  $\DS \lim_{\ntoinf}x_n=a \implies \lim_{\ntoinf}f(x_n)=f(a)$
  \end{enumerate}
\end{definition}
\bigskip

Непрерывные в точке $a$ функции можно складывать, вычитать, умножать и (если $g(a)\ne0$) делить.
\bigbreak

Следствия: \begin{enumerate}
	\item Многочлены $\DS p(x)=\sum_{k=0}^n a_kx^k$ непрерывны на $\RR$.
	\item Отношения многочленов $\DS \frac{p(x)}{q(x)}$ (рациональные функции)
		  непрерывны во всех точках, в которых знаменатель не обращается в ноль.
\end{enumerate}
\bigskip

\begin{theorem}[Теорема о стабилизации знака]
	Если $f\on E\to\RR$ непрерывна в точке $a \in E$ и $f(a)\ne0$,
	то найдётся такая окрестность $U_a$, что знак $f(x)$ совпадает с $f(a)$.
\end{theorem}
\begin{proof} Следствие утверждения про пределы \end{proof}
\bigskip

\begin{theorem}[Непрерывность композиции]
	\lets $f\on D\to\RR, \. g\on E\to\RR \. f(D)\subset E$ и
	$f$ непрерывна в точке $a\in D$, а $g$ непрерывна в точке $f(a)$.
	Тогда $g \circ f$ непрерывна в точке $a$.
\end{theorem}
\begin{proof}
	Доказывается по определению предела по Гейне. \\
	Для пределов было бы неверно, т.к. композиция пределов
	не обязательно будет пределом композиции, напр. для
	$f(x)=x\sin\frac1x$ ($x\ne0$), $g(y)=\{1\Leftarrow y\ne 0 ,\. 0\Leftarrow y=0\}$
\end{proof}
\bigskip

Пример: $\sin x$ непрерывен на $\RR$. \\
Вспомогательное нер-во: если $0<x<\frac\pi2$, то $\sin x<x<\tg x$.

%%%%%%%%%%%%%%%%%%%%%%%%%%%%%%%%%%%%%%%%%%%%%%%%%%%%%
\subsection{Теорема Вейерштрасса}

\begin{theorem}[Вейерштрасса]
  Непрерывная на \textit{отрезке} функция: \\
	\circleone ограничена;
	\circletwo принимает наибольшее и наименьшее значения
\end{theorem}
\begin{proof}
	\circleone от противного:
	\lets неограничена $\Rightarrow$ можно найти $\{x_n\}$ т.ч. $f(x_n)\to+\infty$.
	Т.к. $\{x_n\}$ -- ограничена $\Rightarrow$ по т.Больцано-Вейерштрасса
	найдётся $x_{n_k} \to c$ ... \\
	\circletwo от противного:
	\lets $f(x)<M$. Рассмотреть $g(x)=\frac1{M-f(x)}$ ...
\end{proof}
\bigskip

Расширение теоремы: если функция $f$ непрерывна на $[a,+\infty]$,
и $\exists$ конечный предел $\DS \lim_{x\to+\infty}f(x)$,
то $f$ ограничена на $[a,+\infty]$.


%%%%%%%%%%%%%%%%%%%%%%%%%%%%%%%%%%%%%%%%%%%%%%%%%%%%%
\subsection{Теорема Больцано-Коши}

\begin{theorem}[Больцано-Коши]
	\lets $f$ непрерывна на $[a,b]$ и значения $f(a)$ и $f(b)$ разных знаков.
	Тогда $\exists \, c \in (a,b)$, для которой $f(c)=0$.
\end{theorem}
\begin{proof} Делением отрезка пополам \end{proof}

\begin{corollary}
	\lets $f$ непрерывна на $[a,b]$ и $f(a)<y<f(b)$ или $f(b)<y<f(a)$.
	Тогда $\exists$ такая точка $c\in(a,b)$, что $f(c)=y$.
\end{corollary}
\begin{proof} Рассмотреть $g(x)=f(x)-y$ \end{proof}

\begin{corollary}
	Если \textit{непрерывная} на \textit{отрезке} функция принимает какие-то два значения,
	то она принимает и все значения, лежащие между ними.
\end{corollary}
\bigskip

\begin{corollary}Непрерывный образ отрезка -- отрезок\end{corollary}
\begin{proof} Теорема Вейерштрасса + теорема Больцано-Коши \end{proof}
\bigskip

\begin{theorem}
	Непрерывный образ промежутка -- промежуток (возможно, другого типа).
\end{theorem}
\begin{proof}
	Показать, что значения функции $(m,M) \subset f\left(\intab\right) \subset [m,M]$,
	где $\DS m=\inf_{x\in\intab}f(x)$, $\DS M=\sup_{x\in\intab}f(x)$
\end{proof}
\bigskip

\CenterFigure{\linewidth}{bolcano-koshi-test-1.png}

\subsubsection{Обратные функции}

\begin{definition}[Обратная функция]
	\lets $f\on X\to Y$, причём\begin{enumerate}
		\item $f(x_1)\ne f(x_2)$ при $x_1\ne x_2$ (инъекция)
		\item Для \any $y\in Y$ найдётся такой $x\in X$, что $y=f(x)$ (сюръекция)
	\end{enumerate}
	Тогда можно определить $g\on Y\to X$ так, что\begin{itemize}
		\item $g(f(x))=x$ при всех $x\in X$
		\item $f(g(y))=y$ при всех $y\in Y$
	\end{itemize}
\end{definition}

\begin{theorem}
	\lets $f\on \intab\to\RR$ непрерывна и строго монотонна,
	$\DS m=\inf_{x \in \intab}f(x)$,
	$\DS M=\sup_{x \in \intab}f(x)$.
	Тогда:
	\begin{enumerate}
		\item $f$ -- обратима и $f^{-1}\on \left<m,M\right> \to \intab$
		\item $f^{-1}$ -- строго монотонна
		\item $f^{-1}$ -- непрерывна на $\left<m,M\right>$
	\end{enumerate}
\end{theorem}
\begin{proof}
  \begin{enumerate}
	\item Показать, что $f$ инъективна и сюръективна.
	\item Монотонность $f^{-1}$ -- от противного.
	\item Непрерывность $f^{-1}$ -- \href{https://stepik.org/lesson/9530/step/6}{показать},
		  что в \any точке левый и правый пределы равны.
  \end{enumerate}
\end{proof}
\bigskip

Обратные тригонометрические функции:
\begin{align*}
	  & \arcsin:     \quad (\nearrow) \quad \left[ -1, 1 \right] \to \left[ -\frac\pi2, \frac\pi2 \right]
	\\& \arccos:     \quad (\searrow) \quad \left[ -1, 1 \right] \to \left[ 0, \pi \right]
	\\& \arctg: \;\; \quad (\nearrow) \quad \RR \to \left( -\frac\pi2, \frac\pi2 \right)
	\\& \arcctg:  \, \quad (\searrow) \quad \RR \to \left( 0, \pi \right)
\end{align*}

\bigskip
\begin{problem}Доказать, что многочлен нечётной степени всегда имеет корень\end{problem}
\begin{proof}Показать, что при $x\to\pm\infty$ он принимает значения как $>0$, так и $<0$\end{proof}


%%%%%%%%%%%%%%%%%%%%%%%%%%%%%%%%%%%%%%%%%%%%%%%%%%%%%
\subsection{Замечательные пределы}

\begin{align*}
	   \lim_{x\to 0} \frac{\sin x}x &= 1
	\\ \lim_{x\to 0} \left(x+1\right)^{1/x} &= e
	\\ \lim_{x\to 0} \frac{\ln(1+x)}x &= 1
	\\ \lim_{x\to 0} \frac{e^x-1}x &= 1
	\\ \lim_{x\to 0} \frac{a^x-1}x &= \ln a    &(a>0)
	\\ \lim_{x\to 0} \frac{(1+x)^p-1}x &= p   &(p \ne 0)
\end{align*}


%%%%%%%%%%%%%%%%%%%%%%%%%%%%%%%%%%%%%%%%%%%%%%%%%%%%%
\subsection{Эквивалентные функции}

\begin{definition}[$f \sim g$ при $x \to a$]
	\lets $f,g\on E\to \RR$, $a$ -- предельная точка $E$. \\
	Если $\exists$ такая окрестность $\ringU_a$ и функция $\varphi\on E\to\RR$,
	что $f(x)=\varphi(x)g(x)$ при всех $x\in\ringU_a\cap E$
	и $\DS \lim_{x\to a}\varphi(x)=1$,\\
	то $f \sim g$ при $x\to a$ (или $\DS f \underset{x\to a}{\sim} g$).
\end{definition}

\medskip
Свойства эквивалентности (при $x \to a$): \begin{enumerate}
	\item $f \sim f$
	\item если $f \sim g$, то $g \sim f$
	\item если $f \sim g$ и $g \sim h$, то $f \sim h$
	\item если $f_1 \sim g_1$ и $f_2 \sim g_2$, то $f_1 f_2 \sim g_1 g_2$
	\item если $f_1 \sim g_1$, $f_2 \sim g_2$, причём
		  $f_2(x)\ne0$ и $g_2(x)\ne0$ при $x\in \ringU_a$,
		  то $\DS \frac{f_1}{f_2} \sim \frac{g_1}{g_2}$
\end{enumerate}

Замечание: эквивалентности нельзя складывать, т.е. $f_1 \pm f_2 \not\sim g_1 \pm g_2$

\bigskip
Замечательные пределы ($x\to0$):
\begin{align*}
	   & x \sim \sin x \sim \tg x \sim \arcsin x \sim \arctg x
	\\ & x \sim \ln(1+x)
	\\ & 1 - \cos x \sim \frac{x^2}2
	\\ & a^x-1 \sim x \ln a		& (a>0, a\ne1)
	\\ & (1+x)^p \sim px		& (p\ne0)
\end{align*}


\subsubsection{o-малое}

\begin{definition}[$o(f)$]
	\lets $f,g\on E\to\RR$, $a$ -- предельная точка $E$.
	Если $\exists$ окрестность $\ringU_a$ и функция $\varphi\on E\to\RR$,
	такая что $f(x)=\varphi(x)g(x)$ при всех $x\in\ringU_a\cap E$
	и $\DS \lim_{x\to a}\varphi(x)=0$,\\
	то $f = o(g)$ при $x\to a$ или $\DS f \underset{x\to a}{=} o(g)$
\end{definition}

\medskip
Пример: $\DS f=o(1) \quad \Longleftrightarrow \quad \lim_{x\to a}f(x)=0$
\medskip

\begin{corollary}
	Следующие утверждения равносильны: \begin{enumerate}
		\item $\DS f \underset{x\to a}\sim g$
		\item $f=g+o(g)$ при $x\to a$
		\item $f=g+o(f)$ при $x\to a$
	\end{enumerate}
\end{corollary}

\medskip
Свойства:
\begin{enumerate}
	\item $f \cdot o(g) = o(f \cdot g)$
	\item Если $f$ ограничена в $\ringU_a$, то $o(fg)=o(g)$
	\item $o(g) \pm o(g) = o(g)$ \medspace
		  {\small (т.к. \textit{здесь} равенство означает принадлежность к классу функций)}
	\item Если $f \sim g$, то $o(f)=o(g)$
\end{enumerate}

\medskip
Замечательные пределы:
\begin{align*}
	   \sin x &= x + o(x)
	\\ \cos x &= 1 - \frac{x^2}2 + o(x^2)
	\\ a^x &= 1 + x \ln a + o(x)
	\\ (1+x)^p &= 1 + px + o(x)
\end{align*}

\CenterFigure{\linewidth}{small-o-test-1.png}


%%%%%%%%%%%%%%%%%%%%%%%%%%%%%%%%%%%%%%%%%%%%%%%%%%%%%%%%%%%%%%%%%%%%%%%
%%%%%%%%%%%%%%%%%%%%%%%%%%%%%%%%%%%%%%%%%%%%%%%%%%%%%%%%%%%%%%%%%%%%%%%
\vignette
\section{Производные}


%%%%%%%%%%%%%%%%%%%%%%%%%%%%%%%%%%%%%%%%%%%%%%%%%%%%%
\subsection{Дифференцируемость и производная}


\begin{definition}[Дифференцируемые функции]
	\lets $f\on\intab\to\RR$ и $x_0\in\intab$. \\
	$f$ -- дифференцируема в точке $x_0$, если $\exists$ такое $k\in\RR$, что
	\[ f(x) = f(x_0) + k(x-x_0) + o(x-x_0) \quad при x \to x_0. \]
\end{definition}


\begin{definition}
	Производной функции $f$ в точке $x_0$ называется
	\[ f'(x_0) = \lim_{x\to x_0}\frac{f(x)-f(x_0)}{x-x_0}
			   = \lim_{h\to 0}\frac{f(x_0+h)-f(x_0)}h \]
\end{definition}
\bigskip


\begin{theorem}[Критерий дифференцируемости] \ \\
	\lets $f\on \intab\to\RR$ и $x_0\in\intab$.
	Тогда след. утверждения эквивалентны:
	\begin{enumerate}
		\item $f$ дифференцируема в точке $x_0$
		\item $f$ имеет в точке $x_0$ конечную производную
		\item $\exists$ непрерывная в $x_0$ функция $\varphi\on\intab\to\RR$
			  такая, что $f(x)-f(x_0) = \varphi(x)(x-x_0)$
	\end{enumerate}
	И если они справедливы, то $k=f'(x_0)=\varphi(x_0)$
\end{theorem}
\begin{proof}
	\circletwo $\Rightarrow$ \circlethree: \[
		\varphi(x) = \begin{cases}\begin{aligned}
			& \frac{f(x)-f(x_0)}{x-x_0}  & \text{при } x \ne x_0 \\
			& f'(x_0)                    & \text{при } x=x_0
		\end{aligned}
		\end{cases}
	\]
	\circlethree $\Rightarrow$ \circleone: \[
		f(x) = f(x_0) + \varphi(x_0)(x-x_0) + \left( \varphi(x)-\varphi(x_0) \right) (x-x_0)
	\]
\end{proof}

\begin{corollary}
	$f$ дифференцируема в точке $x_0$ $\implies$ $f$ непрерывна в точке $x_0$
\end{corollary}

\bigskip\medskip


Бесконечные производные:
$\DS \lim_{x\to x_0}\frac{f(x)-f(x_0)}{x-x_0} = \pm\infty$

Односторонние производные:
$\DS f'_{\pm}(x_0) = \lim_{x\to x_0\pm}\frac{f(x)-f(x_0)}{x-x_0}$

Производные можно складывать, вычитать, умножать
и (если знаменатель $\ne0$ в точке $x_0$) делить.
\bigskip


\begin{theorem}[Производная композиции]
	\def\intcd{\left<c,d\right>}
	\lets $f\on \intab \to \intcd$,	\medspace $g\on \intcd \to \RR$,
	\\ $f$ -- дифф. в точке $x_0 \in\intab$, \medspace $g$ -- дифф. в точке $f(x_0)$.
	\\ Тогда $g \circ f$ дифференцируема в $x_0$
	и $(g \circ f)'(x_0) = g'(f(x_0)) \cdot f'(x_0)$
\end{theorem}
\begin{proof}
	Использовать определение дифференцируемости
	\circlethree через непрерывную функцию
\end{proof}
\medskip


\begin{theorem}[Дифференцирование обратной функции]
	\lets $f\on\intab\to\RR$ строго монотонна и непрерывна,
	дифференцируема в точке $x_0 \in\intab$ и $f'(x_0) \ne 0$. \\
	Тогда обратная функция $f^{-1}$ дифф. в точке $f(x_0)$ и
	$\DS \left(f^{-1}\right)'\left(f(x_0)\right) = \frac1{f'(x_0)}$
\end{theorem}


%%%%%%%%%%%%%%%%%%%%%%%%%%%%%%%%%%%%%%%%%%%%%%%%%%%%%
\subsection{Теоремы о среднем}

\begin{theorem}[Теорема Ферма]
	\lets $f\on\intab\to\RR$ дифференцируема в $x_0\in(a,b)$.
	Если $\DS f(x_0)=\max_{x\in\intab}f(x)$ или $f(x_0)=\min_{x\in\intab}f(x)$,
	то $f'(x_0)=0$.
\end{theorem}
\bigskip


\begin{theorem}[Теорема Ролля]
	\lets $f\on[a,b]\to\RR$ непрерывна на $[a,b]$ и дифф. на $(a,b)$.
	Если $f(a)=f(b)$, то $\exists$ такая точка $c\in(a,b)$, что $f'(c)=0$.
\end{theorem}
\bigskip


\begin{theorem}[Теорема Лагранжа]
	\lets $f\on[a,b]\to\RR$ непрерывна на $[a,b]$ и дифф. на $(a,b)$.
	Тогда $\exists$ такая точка $c\in(a,b)$, что $f(b)-f(a)=f'(c)(b-a)$.
\end{theorem}
\bigskip


\begin{theorem}[Теорема Коши]
	...
\end{theorem}

\todo


%%%%%%%%%%%%%%%%%%%%%%%%%%%%%%%%%%%%%%%%%%%%%%%%%%%%%
\subsection{Производная и монотонность}


%%%%%%%%%%%%%%%%%%%%%%%%%%%%%%%%%%%%%%%%%%%%%%%%%%%%%
\subsection{Правило Лопиталя}


%%%%%%%%%%%%%%%%%%%%%%%%%%%%%%%%%%%%%%%%%%%%%%%%%%%%%
\subsection{Формула Тейлора}


%%%%%%%%%%%%%%%%%%%%%%%%%%%%%%%%%%%%%%%%%%%%%%%%%%%%%
\subsection{Экстремумы функций}




%%%%%%%%%%%%%%%%%%%%%%%%%%%%%%%%%%%%%%%%%%%%%%%%%%%%%%%%%%%%%%%%%%%%%%%
%%%%%%%%%%%%%%%%%%%%%%%%%%%%%%%%%%%%%%%%%%%%%%%%%%%%%%%%%%%%%%%%%%%%%%%
\vignette
\section{Интегралы}


%%%%%%%%%%%%%%%%%%%%%%%%%%%%%%%%%%%%%%%%%%%%%%%%%%%%%
\subsection{Первообразная и неопределённый интеграл}

\todo


%%%%%%%%%%%%%%%%%%%%%%%%%%%%%%%%%%%%%%%%%%%%%%%%%%%%%
\subsection{Действия с неопределёнными интегралами}


%%%%%%%%%%%%%%%%%%%%%%%%%%%%%%%%%%%%%%%%%%%%%%%%%%%%%
\subsection{Площади и определённый интеграл}


%%%%%%%%%%%%%%%%%%%%%%%%%%%%%%%%%%%%%%%%%%%%%%%%%%%%%
\subsection{Теорема Барроу и формула Ньютона-Лейбница}


%%%%%%%%%%%%%%%%%%%%%%%%%%%%%%%%%%%%%%%%%%%%%%%%%%%%%
\subsection{Интегральные суммы}


%%%%%%%%%%%%%%%%%%%%%%%%%%%%%%%%%%%%%%%%%%%%%%%%%%%%%
\subsection{Связь между суммами и интегралами}

\vignette

\end{document}
