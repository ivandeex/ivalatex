\documentclass[a4paper,12pt]{article}

%%% Работа с русским языком
\usepackage{cmap}					% поиск в PDF
\usepackage{mathtext} 				% русские буквы в формулах
\usepackage[T2A]{fontenc}			% кодировка
\usepackage[utf8]{inputenc}			% кодировка исходного текста
\usepackage[english,russian]{babel}	% локализация и переносы

%%% Дополнительная работа с математикой
\usepackage{amsfonts,amssymb,amsthm,mathtools} % AMS
\usepackage{amsmath}
\usepackage{icomma} % "Умная" запятая: $0,2$ --- число, $0, 2$ --- перечисление

%% Шрифты
\usepackage{euscript} % Шрифт Евклид
\usepackage{mathrsfs} % Красивый матшрифт

%% Перенос знаков в формулах (по Львовскому)
%\newcommand*{\hm}[1]{#1\nobreak\discretionary{}{\hbox{$\mathsurround=0pt #1$}}{}}

%%% Работа с таблицами
\usepackage{array,tabularx,tabulary,booktabs} % Дополнительная работа с таблицами
\usepackage{longtable}  % Длинные таблицы
\usepackage{multirow} % Слияние строк в таблице

\usepackage{color}

%%% Заголовок
\title{Основы теории графов}
\author{}
\date{}

\begin{document}

\newtheorem{theorem}{Теорема}
\def\ilet{$\gimel$}
\def\iiff{$\;\Longleftrightarrow\;$}
\maketitle

\section{Вершинная связность}

\begin{theorem}[Хёринга]
	$\max$ кол-во путей $P(x \rightarrow y)$ (не перес. во внутр. точках) $=$ $|R|$ -- $\max$ мн-ва вершин, отделяющих $x$ и $y$.
\end{theorem}

\begin{theorem}[Менгера]
	Для $\forall$ несмежных вершин $x,y \in V$ $\nexists e(x,y)$ размер мин. \textbf{верш.}-разделяющего мн-ва $|R_{min}(x \leftrightarrow y)|$ $=$ $\max$ числу простых путей $P(x \rightarrow y)$, отличных во внутренних точках.
\end{theorem}

\begin{theorem}[Уитни]
	Граф $G$ $k$-связный \iiff $\forall x,y \in V$, $\exists$ $k$ простых путей $P(x \rightarrow y)$, не пересекающихся во внутренних точках $P_i \neq P_j \text{(внут.)}$.
\end{theorem}

\section{Рёберная связность}

\begin{theorem}[Форда-Фалкерсона]
$\max$ поток $Q$ через сеть $=$ пропускной способности минимального $S$-$T$ разреза.
\end{theorem}

\begin{theorem}[Менгера ``рёберная'']
Для $\forall$ несмежных вершин $x,y \in V$ $\nexists e(x,y)$ размер $\min$ \textbf{рёберно}-разделяющего мн-ва $|R^{edge}_{min}(x \leftrightarrow y)|=$ $\max$ числу простых \textbf{рёберно}-непересекающихся путей $P(x \rightarrow y)$.
\end{theorem}

\section{Паросочетания}

\begin{center}
\begin{tabular}{|c|c|c|c|}
\cline{2-3}
\multicolumn{1}{c|}{} & вершинное & рёберное     & \multicolumn{1}{|c}{} \\ \hline
незав. мн-во          & $\alpha$  & $\alpha'$    & $\max$                \\ \hline
покрытие              & $\beta$   & $\beta'$     & $\min$                \\ \hline
\multicolumn{1}{c|}{} & вершинное & п-сочетание  & \multicolumn{1}{|c}{} \\ \cline{2-3}
\end{tabular}
\end{center}

\begin{theorem}[Галаи]
\[ \alpha + \beta = \alpha' + \beta' = n \]
\end{theorem}

\begin{theorem}[Кёнига]
В $\forall$ 2-дольном графе $B(m,n)$: $\beta = \alpha'$
\end{theorem}

\begin{theorem}[Татта]
В графе $\exists$ совершенное паро-сочетание \iiff при удалении $\forall$ подмножества вершин $S$ образуется нечётных компонент $C_o(G \setminus S) \leqslant |S|$
\end{theorem}

\end{document}
